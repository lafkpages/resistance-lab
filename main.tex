\documentclass{article}

% For positioning figures (eg. images) with [H]
\usepackage{float}

% Used for \SI and SI units
\usepackage{siunitx}

% Fancy tables innit
% https://tablesgenerator.com/latex_tables
\usepackage{booktabs}

% For \abs
\usepackage{mathtools}
\DeclarePairedDelimiter\abs{\lvert}{\rvert}%

% Swap the definition of \abs* so that it resizes the size of the
% brackets, and the starred version does not.
\makeatletter
\let\oldabs\abs
\def\abs{\@ifstar{\oldabs}{\oldabs*}}
\makeatother

% For including graphs via \includegraphics
\usepackage{graphicx}
\graphicspath{{assets}}

% Graph shortcuts
\newcommand{\graph}[2]{
  \begin{figure}[H]
    \medskip
    \centering
    \includegraphics[width=1\linewidth]{#1}
    \caption{#2}
    \medskip\label{fig:#1}
  \end{figure}
}

% Links
\usepackage{hyperref}
\hypersetup{colorlinks=true, linkcolor=blue, urlcolor=cyan, citecolor=blue}
\urlstyle{same}

\author{Luis F.}
\title{Wire Resistance Lab}

\begin{document}

\maketitle

\section{Lab Design}
\subsection{Research Question}

\begin{itemize}
  \item Research question: How is the resistance of a wire dependent
    on its cross-sectional area?
  \item Independent variable: Cross-sectional area of the wire
  \item Dependent variable: Resistance of the wire
\end{itemize}

The hypothesis is that the resistance of a wire is inversely
proportional to its cross-sectional area. This can be seen in
equation~\ref{eq:resistance}, the equation for resistance. As the
cross-sectional area, \(A\), increases, the resistance, \(R\), decreases.

\begin{equation} \label{eq:resistance}
  R = \frac{\rho L}{A}
\end{equation}

\subsection{Controlled Variables}

The dependent variable, resistance, will be measured using a multimeter. The two probes

\begin{table}[H]
  \centering
  \begin{tabular}{@{}ccc@{}}
    \toprule
    Wire Diameter (mm) & Wire Cross-Sectional Area (mm) & Wire Resistivity \\ \midrule
    1.00               & 0.79                           & \num{1.08e-6}    \\
    0.80               & 0.50                           & \num{1.08e-6}    \\
    0.63               & 0.31                           & \num{1.08e-6}    \\
    0.49               & 0.19                           & \num{1.08e-6}    \\ \bottomrule
  \end{tabular}
  \caption{My table}
  \label{tab:my-table}
\end{table}

\end{document}
